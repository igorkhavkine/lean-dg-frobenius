% In this file you should put the actual content of the blueprint.
% It will be used both by the web and the print version.
% It should *not* include the \begin{document}
%
% If you want to split the blueprint content into several files then
% the current file can be a simple sequence of \input. Otherwise It
% can start with a \section or \chapter for instance.

\chapter{Introduction}

This project aims to really formalize the Frobenius Theorem from differential geometry in Lean 4.

\chapter{Frobenius Theorem}

Consider a smooth manifold $M$ of dimension $m$. For local questions we can take
$M=\mathbb{R}^m$, which could correspond to a chart around some point $x\in M$.
All functions, vector fields and differential forms are presumed to be smooth ($C^\infty$).

\begin{definition}[involutivity] \label{def:invol}
Let $L_i = \sum_{k=1}^m f_i^k(x) \partial/\partial x^j$, $i=1,\ldots,r\le m$, be
first order differential operators, such that the vector fields $v_i(x) =
(f_i^k(x))_{k=1}^m$ are linearly independent. They are said to be in
\emph{involution} when there exist functions $c^k_{ij}(x)$ such that
\[
  L_i L_j - L_j L_i = \sum_{k=1}^r c^k_{ij}(x) L_k .
\]
\end{definition}

\begin{theorem}[local Frobenius] \label{thm:frob-loc}
\uses{def:invol}
If the first order differential operators $L_i$, $i=1,\ldots,r \le m$, are in
involution, then there exist $m-r$ smooth functions $u^k(x)$ that satisfy the
equations $L_i u^k(x) = 0$ and such that their gradients $\nabla u^k(x)$,
$k=1,\ldots,m-r$ are linearly independent.
\end{theorem}

\begin{definition}[differential forms] \label{def:forms}
\notready
\end{definition}

\begin{definition}[differential ideal] \label{def:diff-ideal}
\uses{def:forms}
\notready
\end{definition}

\begin{theorem}[differential form Frobenius] \label{thm:frob-forms}
\uses{def:diff-ideal, thm:frob-loc}
If $\alpha_i$, $i=1,\ldots k\le m-k$ are 1-forms on $M$ that generate a closed
differential ideal. Then there exist smooth scalar functions $u_i(x)$,
$i=1,\ldots,m-k$ such that the exact 1-forms $du_i$, $i=1,\ldots,m-k$ generate
the same differential ideal.
\end{theorem}

\begin{definition}[tangent distribution] \label{def:vec-dist}
A \emph{tangent distribution} on a manifold $M$ is a vector sub-bundle
$\mathcal{D}\hookrightarrow TM$ (equivalently, an embedding of vector bundles).
\end{definition}

\begin{definition}[Lie bracket] \label{def:lie}
\uses{def:invol}
On a manifold $M$, given two vector fields $u$, $v$ (sections of the tangent
bundle $TM$), their \emph{Lie bracket} $w = [u,v]$ is the vector field that
satisfies the identity $w(f) = u(v(f)) - v(u(f))$, where vector fields act as
first order differential operators on a smooth function $f$. In coordinate form,
if $u = u^i\partial_i$, $v = v^i\partial_i$, $w = w^i\partial_i$, then $w^j =
u^i \partial_i v^j - v^i \partial_i u^i$. The vector fields $u$, $v$
\emph{commute} (or are \emph{in involution} in the sense of
Def.~\ref{def:invol}) if $[u,v] = 0$.
\end{definition}

\begin{definition}[involutive distribution] \label{def:inv-distr}
\uses{def:lie,def:vec-dist}
A tangent distribution $\mathcal{D} \hookrightarrow TM$ is \emph{involutive} if,
for any two vector field sections $u$, $v$ of $\mathcal{D}$, the Lie bracket
$[u,v]$ is also a section of $\mathcal{D}$.
\end{definition}

\begin{definition}[integral submanifold] \label{def:int-subman}
\uses{def:vec-dist}
Given a manifold $M$ with a tangent distribution $\mathcal{D} \hookrightarrow
TM$ of rank $r$ (as a vector bundle), a submanifold $\iota\colon N
\hookrightarrow M$ passing through $x_0 \in M$ is called an \emph{integral
submanifold} of the distribution $\mathcal{D}$ if it is everywhere tangent to
$\mathcal{D}$, $T\iota (TN) \subseteq \mathcal{D}$, where naturally $\dim N \le
r$. In the case $\dim N = r$, the integral submanifold is called \emph{maximal
(in dimension)}.
\end{definition}

\begin{definition}[foliation] \label{def:foliation}
\notready
\end{definition}

\begin{theorem}[vector field Frobenius] \label{thm:frob-vec}
\uses{def:inv-distr, def:int-subman, def:foliation, thm:frob-loc}
Let $\mathcal{D} \subseteq TM$ be an involutive tangent space distribution of
rank $r\le m = \dim M$. Then, for every $x\in M$, there exists a maximal
integral submanifold $\iota\colon \mathbb{R}^n \hookrightarrow M$ of
$\mathcal{D}$ such that $\iota(0) = x$. Moreover, these integral submanifolds
collect into a $r$-dimensional foliation of $M$ whose leaves are everywhere
tangent to the distribution $\mathcal{D}$.  Then there exists a foliation of $M$
with $r$-dimensional leaves that are everywhere tangent to the distribution
$\mathcal{D}$.
\end{theorem}
