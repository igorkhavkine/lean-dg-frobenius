% In this file you should put the actual content of the blueprint.
% It will be used both by the web and the print version.
% It should *not* include the \begin{document}
%
% If you want to split the blueprint content into several files then
% the current file can be a simple sequence of \input. Otherwise It
% can start with a \section or \chapter for instance.

\chapter{Introduction}

This project aims to really formalize the Frobenius Theorem from differential geometry in Lean 4.

\chapter{Frobenius Theorem}

Consider a smooth manifold $M$ of dimension $m$. For local questions we can take
$M=\mathbb{R}^m$, which could correspond to a chart around some point $x\in M$.
All functions, vector fields and differential forms are presumed to be smooth ($C^\infty$).

\begin{definition}[involutivity] \label{def:invol}
Let $L_i = \sum_{k=1}^m f_i^k(x) \partial/\partial x^j$, $i=1,\ldots,r\le m$, be
first order differential operators, such that the vector fields $v_i(x) =
(f_i^k(x))_{k=1}^m$ are linearly independent. They are said to be in
\emph{involution} when there exist functions $c^k_{ij}(x)$ such that
\[
  L_i L_j - L_j L_i = \sum_{k=1}^r c^k_{ij}(x) L_k .
\]
\end{definition}

\begin{theorem}[local Frobenius] \label{thm:frob-loc}
\uses{def:invol}
If the first order differential operators $L_i$, $i=1,\ldots,r \le m$, are in
involution, then there exist $m-r$ smooth functions $u^k(x)$ that satisfy the
equations $L_i u^k(x) = 0$ and such that their gradients $\nabla u^k(x)$,
$k=1,\ldots,m-r$ are linearly independent.
\end{theorem}

\begin{definition}[differential forms] \label{def:forms}
\notready
\end{definition}

\begin{definition}[differential ideal] \label{def:diff-ideal}
\uses{def:forms}
\notready
\end{definition}

\begin{theorem}[differential form Frobenius] \label{thm:frob-forms}
\uses{def:diff-ideal, thm:frob-loc}
If $\alpha_i$, $i=1,\ldots r\le m-r$ are 1-forms on $M$ that generate a closed
differential ideal. Then there exist smooth scalar functions $u_i(x)$,
$i=1,\ldots,m-r$ such that the exact 1-forms $du_i$, $i=1,\ldots,m-r$ generate
the same differential ideal.
\end{theorem}

\begin{definition}[vector field distribution] \label{def:vec-dist}
\notready
\end{definition}

\begin{definition}[involutive distribution] \label{def:inv-distr}
\uses{def:vec-dist}
\notready
\end{definition}

\begin{theorem}[vector field Frobenius] \label{thm:frob-vec}
\uses{def:inv-distr, thm:frob-loc}
Let $\mathcal{V} \subseteq TM$ be an involutive tangent space distribution of
rank $r\le m$. Then there exists a foliation of $M$ with $r$-dimensional leaves
that are everywhere tangent to the distribution $\mathcal{V}$.
\end{theorem}
