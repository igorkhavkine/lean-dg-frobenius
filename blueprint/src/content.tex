% In this file you should put the actual content of the blueprint.
% It will be used both by the web and the print version.
% It should *not* include the \begin{document}
%
% If you want to split the blueprint content into several files then
% the current file can be a simple sequence of \input. Otherwise It
% can start with a \section or \chapter for instance.

\chapter{Introduction}

This project aims to really formalize the Frobenius Theorem from differential geometry in Lean 4.

\chapter{Frobenius Theorem}

Consider a smooth manifold $M$ of dimension $m$. For local questions we can take
$M=\mathbb{R}^m$, which could correspond to a chart around some point $x_0\in M$.
All functions, vector fields and differential forms are presumed to be smooth ($C^\infty$).

\begin{definition}[involutivity] \label{def:invol}
Let $L_i = \sum_{k=1}^m f_i^k(x) \partial/\partial x^j$, $i=1,\ldots,r\le m$, be
first order differential operators, such that the vector fields $v_i(x) =
(f_i^k(x))_{k=1}^m$ are linearly independent. They are said to be in
\emph{involution} when there exist functions $c^k_{ij}(x)$ such that
\[
  L_i L_j - L_j L_i = \sum_{k=1}^r c^k_{ij}(x) L_k .
\]
\end{definition}

\begin{theorem}[local Frobenius] \label{thm:frob-loc}
\uses{def:invol,lem:op-frame,lem:eq-frame-equiv,lem:loc-init-split,lem:eq-loc-inv,lem:eq-radial,lem:eq-radial-exists}
If the first order differential operators $L_i$, $i=1,\ldots,r \le m$, are in
involution, then there exist $m-r$ smooth functions $u^k(x)$ that satisfy the
equations $L_i u^k(x) = 0$ and such that their gradients $\nabla u^k(x)$,
$k=1,\ldots,m-r$ are linearly independent.
\end{theorem}

\begin{lemma} \label{lem:op-frame}
If the $L_i$ as in Def.~\ref{def:invol} are in infolution, then the $L'_{i'} =
\sum_{i=1}^r \alpha_{i'}^i L_i$, $i'=1,\ldots,r$, with smooth pointwise invertible
$\alpha_{i'}^i$, are also in involution.
\end{lemma}
\begin{proof}
It is sufficient to compute the commutator
\begin{align*}
  L'_{i'} L'_{j'} - L'_{j'} L'_{i'}
  &= \alpha_{i'}^i \alpha_{j'}^j (L_i L_j - L_j L_i)
    + \alpha_{i'}^i L_i(\alpha_{j'}^j) L_j
    - \alpha_{j'}^j L_j(\alpha_{i'}^i) L_i
  \\
  &= \alpha_{i'}^i \alpha_{j'}^j \left(c_{ij}^k
    + (\alpha^{-1})_j^{j'} L_i(\alpha_{j'}^j) \delta_j^k
    - (\alpha^{-1})_i^{i'} L_j(\alpha_{i'}^i) \delta_i^k \right) L_k
  \\
  &= \alpha_{i'}^i \alpha_{j'}^j \left(c_{ij}^k
    + (\alpha^{-1})_j^{j'} L_i(\alpha_{j'}^j) \delta_j^k
    - (\alpha^{-1})_i^{i'} L_j(\alpha_{i'}^i) \delta_i^k \right) (\alpha^{-1})_k^{k'} L'_{k'}
  \\
  &= \sum_{k'=1}^r c'^{k'}_{i'j'} L'_{k'} ,
\end{align*}
where the formula for $c'^{k'}_{i'j'}$ can be read off from the last equality.
\end{proof}

\begin{lemma} \label{lem:eq-frame-equiv}
For the operators $L'_{i'}$ as in Lem.~\ref{lem:op-frame}, a smooth function $u$
solves $L_i u = 0$, $i=1,\ldots,r$, iff it solves $L'_{i'} u = 0$,
$i'=1,\ldots,r$.
\end{lemma}
\begin{proof}
The computation
\[
  L'_{i'} u = \sum_{i=1}^r \alpha_{i'}^i (L_i u)
\]
shows that $L_i u = 0$, $i=1,\ldots,r$, implies $L'_{i'} u = 0$ for any $i'=1,\ldots,r$.
\end{proof}

\begin{lemma} \label{lem:loc-init-split}
\uses{def:invol,lem:op-frame}
For the operators $L_i$ from Def.~\ref{def:invol}, given $x_0 \in M$, there exists
an open coordinate neighborhood $U\ni x_0$ such that (a) there an exists invertible
$\alpha_{i'}^i$ as in Lem.~\ref{lem:op-frame}, (b) there exists a split local
chart $(y,z)\colon M \supset U' \cong \mathbb{R}^r \times \mathbb{R}^{m-r}$, with
(c) $(y(x_0),z(x_0)) = (0,0)$, (d)
\[
  L'_{i'} = \frac{\partial}{\partial y^{i'}} + \sum_{j=1}^{m-r} f_{i'}^j(y,z) \frac{\partial}{\partial z^j} .
\]
and (e) $[L'_{i'}, L'_{j'}] = 0$, for $i',j'=1,\ldots,r$.
\end{lemma}
\begin{proof}
Start with the coordinates $(x^1,\ldots,x^m)$ on $U$ and consider the coordinate
components $L_i = a_i^j(x) \frac{\partial}{\partial x^j}$. The rank of the
matrix $a_i^j(x_0)$ must be $r$, otherwise the $L_i$ vectors do not constitute a
frame for the distribution $\mathcal{D}$. Hence, there exists a subset $I
\subseteq \{1,\ldots,r\}$ such that the matrix minor $(a_i^j)_{i\in I, 1\le j\le
m}$ is non-singular. Define the coordinates $y^{i'} = x^{I(i')} -
(x_0)^{I(i')}$, $i'=1,\ldots,r$, and $z^{j} = x^{I^c(j)} - (x_0)^{I^c(j)}$,
$j=1,\ldots,m-r$, where $I(i')$ and $I^c(j)$ is some ordering of the sets $I$
and its complement $I^c$. Then, restrict to a sub-neighborhood $U'' \subseteq U$
that is split with respect to the $(y,z)$ coordinates.

The new coordinate components are
\[
  L_{i} = \sum_{i'=1}^r a_{i}^{I(i')}(x(y,z)) \frac{\partial}{\partial y^{i'}}
    + \sum_{j=1}^{m-r} a_{i}^{I^c(j)}(x(y,z)) \frac{\partial}{\partial z^j} .
\]
Let $\beta_{i}^{i'}(y,z) = a_{i}^{I(i')}(x(y,z))$ and $\gamma_{i}^{j}(y,z) =
a_{i}^{I^c(j)}(x(y,z))$, so that by construction $\beta_{i}^{i'}(0,0)$ is
non-singular. Since $\beta \colon U'' \to \operatorname{Mat}(r,r)$ is smooth
(hence a fortiriori continuous) and the subset of non-singular matrices in
$\operatorname{Mat}(r,r)$ is open, there is a possibly smaller split
sub-neighborhood $U' \subseteq U''$ on which $\beta$ is everywhere non-singular.
So, defining $\alpha_{i'}^j(y,z) = (\beta_{i}^{i'}(y,z))^{-1}$ on $U''$
satisfies the desired conclusions (a), (b), (c) and (d), where $f_{i'}^j(y,z) =
\alpha_{i'}^i(y,z) \gamma_{i}^j(y,z)$.

To prove (e), consider the computation
\begin{align*}
  [L'_{i'}, L'_{j'}]
  &= L'_{i'} L'_{j'} - L'_{j'} L'_{i'}
  \\
  = \sum_{k'=1}^r c'^{k'}_{i'j'} L'_{k'}
  &= \sum_{j=1}^{m-r} \left(\frac{\partial}{\partial y^{i'}} f_{j'}^j - \frac{\partial}{\partial y^{j'}} f_{i'}^j\right)\frac{\partial}{\partial z^j}
    + \sum_{k=1}^{m-r} \sum_{j=1}^{m-r} \left(f_{i'}^j \frac{\partial}{\partial z^j} f_{j'}^k - f_{j'}^j \frac{\partial}{\partial z^j} f_{i'}^k\right) \frac{\partial}{\partial z^k} .
\end{align*}
Hence, for each fixed $i',j'$, the $\frac{\partial}{\partial y^{k'}}$ components
of the right-hand side vanish, while those of the left-hand side equal
$\sum_{k'=1}^{m-r} c'^{k'}_{i'j'} \frac{\partial}{\partial y^{k'}}$, meaning
that all components of $c'^{k'}_{i'j'}$ must vanish.
\end{proof}

\begin{lemma} \label{lem:eq-loc-inv}
\uses{lem:loc-init-split}
Consider the operators $L'_{i'}$ and the split neighborhood $U\ni x_0$ as in
Lem.~\ref{lem:loc-init-split}. Let $Z^k(y,z)$ and $\bar{Z}^k(y,z)$ satisfy the
inversion identity $\bar{Z}^k(y, Z(y, z)) = z^k$, for all $z$ on a sufficiently
small neighborhood of $z=0$, and for all $y$ on a sufficiently small
neighborhood of $y=0$. Suppose that $Z^j(y,z)$ satisfies (a) $Z^j(0,z) = z^j$
and (b)
\[
  \frac{\partial Z^j}{\partial y^{i'}}(y,z) = f_{i'}^j(y, Z^j(y,z)) .
\]
Then
\[
  L'_{i'} \bar{Z}(y,z) = 0, \quad i'=1,\ldots,r ,
\]
and vice versa.
\end{lemma}
\begin{proof}
Start by differentiating the inversion identity:
\begin{align*}
  0 = \frac{\partial}{\partial y^{i'}} z^k
  &= \frac{\partial}{\partial y^{i'}} \bar{Z}^k(y, Z(y, z))
  \\
  &= \left. \frac{\partial \bar{Z}^k}{\partial y^{i'}}(y, z') \right|_{z'=Z(y,z)}
    + \frac{\partial Z^j}{\partial y^{i'}}(y,z)
      \left. \frac{\partial \bar{Z}^k}{\partial z'^j}(y,z') \right|_{z'=Z(y,z)}
  \\
  &= \left. \left(\frac{\partial \bar{Z}^k}{\partial y^{i'}}(y, z')
      + f_{i'}^j(y,z') \frac{\partial \bar{Z}^k}{\partial z'^j}(y,z')\right) \right|_{z'=Z(y,z)}
    \\ &\quad {}
    + \left. \left(\frac{\partial Z^j}{\partial y^{i'}}(y,z) - f_{i'}^j(y,z')\right)
      \frac{\partial \bar{Z}^k}{\partial z'^j}(y,z') \right|_{z'=Z(y,z)}
    .
\end{align*}
Recall that being a diffeomorphism, the Jacobian $\frac{\partial
\bar{Z}^k}{\partial z'^j}(y,z')$ non-singular on the sufficiently small split
domain, with $(\frac{\partial
\bar{Z}^k}{\partial z'^j}(y,z'))^{-1} = \left. \frac{\partial Z^j}{\partial z^k}(y,z) \right|_{z=\bar{Z}(y,z')}$. Hence, rearranging the last equality, we find
\[
  \left. \frac{\partial Z^j}{\partial z^k}(y,z)
    L'_{i'} \bar{Z}^k(y,z') \right|_{z'=Z(y,z)}
  = -\left(\frac{\partial Z^j}{\partial y^{i'}}(y,z) - f_{i'}^j(y,Z(y,z))\right) .
\]
Hence, if one side of the equality vanishes, then so does the other, which
proves the desired equivalence.
\end{proof}

\begin{lemma} \label{lem:eq-radial}
\uses{lem:eq-loc-inv}
Let $Z^j(y,z)$ be as in Lem.~\ref{lem:eq-loc-inv}. Then, $\zeta^j(t,y,z) =
Z^j(ty,z)$ satisfies $\zeta^j(0,y,z) = z^j$ and the equations
\[
  \frac{\partial}{\partial t} \zeta^j(t,y,z) = y^{i'} f_{i'}^j(ty,\zeta(t,y,z)) .
\]
Conversely, if $\zeta^j(t,y,z)$ satisfies the equation above, then $Z^j(y,z) =
\zeta^j(1,y,z)$, defined on a sufficiently small neighborhood of $y=0$,
satisfies the conditions in the hypotheses of Lem.~\ref{lem:eq-loc-inv}.
\end{lemma}

\begin{lemma} \label{lem:eq-radial-exists}
The ODE initial value problem for $\zeta^j(t,y,z)$ in Lem.~\ref{lem:eq-radial}
(a) has a solution that is jointly smooth in $(t,y,z)$, which is (b) unique.
\end{lemma}

\begin{definition}[differential forms] \label{def:forms}
\notready
\end{definition}

\begin{definition}[differential ideal] \label{def:diff-ideal}
\uses{def:forms}
\notready
\end{definition}

\begin{theorem}[differential form Frobenius] \label{thm:frob-forms}
\uses{def:diff-ideal, thm:frob-loc}
If $\alpha_i$, $i=1,\ldots k\le m-k$ are 1-forms on $M$ that generate a closed
differential ideal. Then there exist smooth scalar functions $u_i(x)$,
$i=1,\ldots,m-k$ such that the exact 1-forms $du_i$, $i=1,\ldots,m-k$ generate
the same differential ideal.
\end{theorem}

\begin{definition}[tangent distribution] \label{def:vec-dist}
\lean{SmoothManifoldWithCorners, VectorBundle, SmoothSection, TangentBundle}
A \emph{tangent distribution} on a manifold $M$ is a vector sub-bundle
$\mathcal{D}\hookrightarrow TM$ (equivalently, an embedding of vector bundles).
\end{definition}

\begin{definition}[Lie bracket] \label{def:lie}
\uses{def:invol}
On a manifold $M$, given two vector fields $u$, $v$ (sections of the tangent
bundle $TM$), their \emph{Lie bracket} $w = [u,v]$ is the vector field that
satisfies the identity $w(f) = u(v(f)) - v(u(f))$, where vector fields act as
first order differential operators on a smooth function $f$. In coordinate form,
if $u = u^i\partial_i$, $v = v^i\partial_i$, $w = w^i\partial_i$, then $w^j =
u^i \partial_i v^j - v^i \partial_i u^i$. The vector fields $u$, $v$
\emph{commute} (or are \emph{in involution} in the sense of
Def.~\ref{def:invol}) if $[u,v] = 0$.
\end{definition}

\begin{definition}[involutive distribution] \label{def:inv-distr}
\uses{def:lie,def:vec-dist}
A tangent distribution $\mathcal{D} \hookrightarrow TM$ is \emph{involutive} if,
for any two vector field sections $u$, $v$ of $\mathcal{D}$, the Lie bracket
$[u,v]$ is also a section of $\mathcal{D}$.
\end{definition}

\begin{definition}[integral submanifold] \label{def:int-subman}
\uses{def:vec-dist}
Given a manifold $M$ with a tangent distribution $\mathcal{D} \hookrightarrow
TM$ of rank $r$ (as a vector bundle), a submanifold $\iota\colon N
\hookrightarrow M$ passing through $x_0 \in M$ is called an \emph{integral
submanifold} of the distribution $\mathcal{D}$ if it is everywhere tangent to
$\mathcal{D}$, $T\iota (TN) \subseteq \mathcal{D}$, where naturally $\dim N \le
r$. In the case $\dim N = r$, the integral submanifold is called \emph{maximal
(in dimension)}.
\end{definition}

\begin{definition}[foliation] \label{def:foliation}
\notready
\end{definition}

\begin{theorem}[vector field Frobenius] \label{thm:frob-vec}
\uses{def:inv-distr, def:int-subman, def:foliation, thm:frob-loc}
Let $\mathcal{D} \subseteq TM$ be an involutive tangent space distribution of
rank $r\le m = \dim M$. Then, for every $x_0\in M$, there exists a maximal
integral submanifold $\iota\colon \mathbb{R}^n \hookrightarrow M$ of
$\mathcal{D}$ such that $\iota(0) = x_0$. Moreover, these integral submanifolds
collect into a $r$-dimensional foliation of $M$ whose leaves are everywhere
tangent to the distribution $\mathcal{D}$.
\end{theorem}
